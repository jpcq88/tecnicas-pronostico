\documentclass[11pt]{article}

\usepackage[utf8]{inputenc}
\usepackage[spanish, es-tabla]{babel}
\spanishdecimal{.}

%\usepackage{booktabs}
\usepackage{mathptmx}
%\usepackage{fontspec}
%\setmainfont{Times New Roman}

\title{Trabajo 1: Análisis IVA Colombia}
\author{Juan Pablo Calle Quintero}

\begin{document}
\maketitle

\section*{Punto 1}

% parámetros ajustados modelo 1
\begin{table}[ht]
\centering
\begin{tabular}{lrrrr}
\hline
 Parámetro & Estimación & Error estándar & Valor $t_0$ & Pr($t_{49}>$$|t_0|$) \\ 
  \hline
$\beta_0$ & 7.4488 & 0.0244 & 305.85 & $<$2$\times10^{-16}$ \\ 
  $\beta_1$ & 0.0471 & 0.0018 & 26.30 & $<$2$\times10^{-16}$ \\ 
  $\beta_2$ & -0.0003 & 3.097$\times10^{-5}$ & -9.18 & 3.14$\times10^{-12}$ \\ 
  $\delta_2$ & 0.0262 & 0.0195 & 1.34 & 0.1857 \\ 
  $\delta_3$ & 0.0394 & 0.0195 & 2.02 & 0.0490 \\ 
  $\delta_4$ & 0.1315 & 0.0199 & 6.60 & 2.78$\times10^{-8}$ \\ 
   \hline
	$AIC=$ 114754.0 & $BIC=$ 142846.8 & & & \\
	\hline
\end{tabular}
\end{table}

% parámetros ajustados modelo 2
\begin{table}[ht]
\centering
\begin{tabular}{lrrrr}
\hline
 Parámetro& Estimación & Error estándar & Valor $t_0$ & Pr($t_{49}>$$|t_0|$) \\ 
  \hline
$\beta_0$ & 7.465 & 0.0486 & 153.529 & $<$2$\times10^{-16}$ \\ 
  $\beta_1$ & 0.0467 & 0.0028 & 16.491 & $<$2$\times10^{-16}$ \\ 
  $\beta_2$ & -0.0003 & 4.038$\times10^{-5}$ & -6.957 & 7.73$\times10^{-9}$ \\ 
  $\delta_2$ & 0.0228 & 0.0204 & 1.116 & 0.2700 \\ 
  $\delta_3$ & 0.0312 & 0.0202 & 1.545 & 0.129 \\ 
  $\delta_4$ & 0.1198 & 0.0200 & 5.985 & 2.47$\times10^{-7}$ \\ 
   \hline
	$AIC=$ 113631.6 & $BIC=$ 141449.6 & & & \\
	\hline
\end{tabular}
\end{table}

% parámetros ajustados modelo 1 sin triemestres 2 y 3
\begin{table}[ht]
\centering
\begin{tabular}{lrrrr}
\hline
 Parámetro & Estimación & Error estándar & Valor $t_0$ & Pr($t_{49}>$$|t_0|$) \\ 
  \hline
$\beta_0$ & 7.470 & 0.0223 & 334.835 & $<$2$\times10^{-16}$ \\ 
  $\beta_1$ & 0.0471 & 0.0018 & 25.771 & $<$2$\times10^{-16}$ \\ 
  $\beta_2$ & -2.845$\times10^{-4}$ & 3.164$\times10^{-5}$ & -8.991 & 4.25$\times10^{-12}$ \\ 
  $\delta_4$ & 0.1096 & 0.0167 & 6.531 & 2.99$\times10^{-8}$ \\ 
   \hline
	$AIC=$ 120338.0 & $BIC=$ 149797.8 & & & \\
	\hline
\end{tabular}
\end{table}

% parámetros ajustados modelo 2 sin trimestres 2 y 3
\begin{table}[ht]
\centering
\begin{tabular}{lrrrr}
\hline
 Parámetro& Estimación & Error estándar & Valor $t_0$ & Pr($t_{49}>$$|t_0|$) \\ 
  \hline
$\beta_0$ & 7.485 & 0.0470 & 159.095 & $<$2$\times10^{-16}$ \\ 
  $\beta_1$ & 0.0465 & 0.0028 & 16.364 & $<$2$\times10^{-16}$ \\ 
  $\beta_2$ & -0.0003 & 4.054$\times10^{-5}$ & -6.851 & 9.37$\times10^{-9}$ \\ 
  $\delta_4$ & 0.1014 & 0.0159 & 6.372 & 5.34$\times10^{-8}$ \\ 
   \hline
	$AIC=$ 119488.8 & $BIC=$ 148740.7 & & & \\
	\hline
\end{tabular}
\end{table}

% resumen LOESS lineal
\begin{table}[ht]
\centering
\begin{tabular}{lr}
 Ajuste LOESS lineal & \\ 
 \hline
 Número de observaciones & 338 \\
 Número equivalente de parámetros & 29.07 \\
 $\sqrt{MSE}$ & 0.099 \\
 Traza matriz de suavizamiento & 34.38\\
Grado polinomio local & 1 (lineal)\\
   \hline
\end{tabular}
\end{table}

% resumen LOESS cuadrático
\begin{table}[ht]
\centering
\begin{tabular}{lr}
 Ajuste LOESS cuadrático & \\ 
 \hline
 Número de observaciones & 338 \\
 Número equivalente de parámetros & 9.38 \\
 $\sqrt{MSE}$ & 0.1721 \\
 Traza matriz de suavizamiento & 10.36\\
Grado polinomio local & 2 (cuadrático)\\
   \hline
\end{tabular}
\end{table}

% pronósticos modelo 1 sin trimestres 2 y 3
\begin{table}[ht]
\centering
\begin{tabular}{lrrrr}
\hline
Periodo & L & Pronóstico & Límite inferior & Límite superior \\
\hline
2014 T4 & 1 & 11223.58 & 9972.457 & 12631.68 \\
2015 T1 & 2 & 10210.27 & 9089.748 & 11468.92 \\
2015 T2 & 3 & 10358.17 & 9208.104 & 11651.88 \\
2015 T3 & 4 & 10502.24 & 9321.256 & 11832.85 \\
\hline
\end{tabular}
\end{table}

% pronósticos modelo 2 sin trimestres 2 y 3
\begin{table}[ht]
\centering
\begin{tabular}{lrr}
\hline
Periodo & L & Pronóstico \\
\hline
2014 T4 & 1 & 11146.55 \\
2015 T1 & 2 & 10225.10 \\
2015 T2 & 3 & 10375.05 \\
2015 T3 & 4 & 10521.36 \\
\hline
\end{tabular}
\end{table}

% medidas d eprecisión modelos 1 y 2 sin trimestre 2 y 3
\begin{table}[ht]
\centering
\begin{tabular}{lrr}
\hline
Medida & Modelo 1 & Modelo 2 \\
\hline
ME & 75.684 & 82.232 \\
RMSE & 397.453 & 381.694 \\
MAE & 386.062 & 362.537 \\
MPE & 0.631 & 0.686 \\
MAPE & 3.632 & 3.420 \\
\hline
\end{tabular}
\end{table}

\end{document}
